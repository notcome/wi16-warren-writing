\documentclass[12pt,letterpaper]{article}

\usepackage{ifpdf}
\usepackage{mla}

\begin{document}
\begin{mla}{Minsheng}{Liu}{Marissa Perino}{WCWP 10A}{03/11/16}
  {3G Reflection}

I have learned a lot of useful stuffs in Warren Writing.

First of all, I realized how to start an essay properly. I should find a
level three question. In the past, I often asked a level two question
and find my whole essay too trivial, or I would write with a level four
question and yield an unconvincing passage. In fact, during the course I
actually had some difficulties in finding the correct questions to ask,
especially in my second and third paper (cf.~2G). I believe I need to
practice more with this, but I am on the right track.

Another very important skill I learned in this course is how to revise
my essay. Thanks to the extensive use of peer reviews, I realized lots
of issues in my essay. Among them the most prominent is readability---I
often think that I have explained something thoroughly, but in fact it
is unintelligible to other people. My peers---and the instructor
Ms.~Perino of course---pointed this out for me for several times.
Without them, my final paper cannot be of that quality.

A good example of this is my discussion of natural language processing
and mind. I think everyone should have read some articles by Chomsky.
People should know the three levels frameworks proposed by David
Marr---the computation level, the algorithmic level, and the
implementation level. However, this could not be further from the truth.
One of my peer pointed out that she did not quite understand the
relationship between our brain and our language system---why the study
of the latter is important to the study of the former. I should be more
careful when writing essays.

I am confident with my logical skills, but writing an essay is more than
proving a proposition. It is about how to spend limited mental resources
of readers rationally to convince them in a most efficient way.
Ms.~Perino has pointed out that while I spent too many words on trivial
parts, I often wrote too little on important facts. For instance, in my
first paper, I stressed too much on the issue of collective action
problems, but it was actually a distraction from my main argument. In my
third paper, I tried to make my three directions distinct and explicitly
mentioned its difference, but it turned out to be unnecessary---a
modification to my structure solved all the problem. In other times,
points were really not well-explained. A good example would be my
discussion of Singer's examples. I \emph{assumed} that readers are as
familiar with topics in my mind as I am, but that is simply impossible.
During my revision, I kept reorganizing my essay so that it was more
reader friendly, but this is a skill that I just started to practice. I
will continue work on this.

In short, I learned a lot in this course. Among them three are most
important: how to find the right topic, how to ensure the readability,
and how to organize my essay.

\end{mla}
\end{document}
