\documentclass[12pt,letterpaper]{article}

\usepackage{ifpdf}
\usepackage{mla}

\begin{document}
\begin{mla}{Minsheng}{Liu}{Marissa Perino}{WCWP 10A}{02/18/16}{Copyright is Necessary}

  Has intellectual property become a concept of the past? Indeed, in the
  age of Internet, sharing without permission has become a \emph{de facto}
  standard, as described by Tomar in his book \emph{Use Me} (85).
  Moreover, other people are arguing that the idea of copyright is wrong
  in the first place because a new piece of knowledge is always created
  from old ones. In his article \emph{Something Borrowed}, Malcolm
  Gladwell gently supported the practice of using ``old words'' to
  establish ``new ideas'' (127). In \emph{RiP!: A Remix Manifesto}, Brett
  Gaylor criticized the copyright laws, showing how the current copyright
  protection institution impeded artists' creativity (Gaylor 2008). I
  partially agree with the view that the domain of copyright should be
  restricted to the commitment part of creative works. More specifically,
  I disagree with patenting ideas. However, I insist that the protection
  of intellectual property is necessary for soliciting creativity. Not
  only does it promote fairness among creators, but it also protects them
  from unintended harm. Simply put, a copyright law that dedicates to the
  devotions of artists is more beneficial than the copyleft ideology.

  Abstract ideas such as writing a novel in the order reverse to its plot
  shouldn't count as intellectual property. Instead, it is the words that
  compose the whole story---the hard work of the author---that should be
  protected. There are two reasons. First, it is hard, if not impossible,
  to find out the inventor of those vague ideas. In \emph{Something
  Borrowed}, Gladwell mentioned that the opera writer Lavery borrowed his
  phrase ``sins and symptoms,'' even though the phrase was a quote from
  Gandhi, who probably learned it from somewhere else (128). Second, these
  ideas are scarce. If they are all owned by some people, others will not
  be able to create works in the same field. For example, there are only a
  few possible plots in spy novels (Widdicombe 52). If they are all
  patented, authors will have to wait until these patents expired before
  writing new spy novels. From this perspective, protecting general ideas
  violates the very principle behind the first copyright law: it should
  give incentive to the creator to create \emph{more} (Gaylor 2008), not
  \emph{less}.

  Why is it important to protect the commitment part of a creative work
  then? It matters because it creates a healthy market in which creators
  compete in a fair way and contribute works of high qualities. A creator
  who has produced an excellent piece of work will likely to have great
  works in the future since the previous work has proven the originator's
  skills. The society rewards these talented artists with wealth and
  reputation as a way to encourage them to create more. Besides, those
  without popular creations will be forced out of the market. All these
  characteristics of the ``intellectual market'' constitute an implicit
  selection process that picks out the best creators. However, declaring
  others' commitments as one own---also known as plagiarism---breaks the
  game rule. In \emph{The Plagiarist's Tale}, Rowan did a ``literature
  mashup''. He wrote a spy novel by copying words from other books
  (Widdicombe 53). Since writers' skills depend on their abilities to put
  words together, by his mashup Rowan tricked readers into thinking that
  he was a capable writer who would produce fantastic novels in the
  future. Plagiarism of this type must be eliminated for three reasons.
  First, unpunished plagiarism like the one of Rowan would encourage
  greedy people to do the same thing. Second, writers who genuinely
  deserve acclaims will gain less from their hardworking. Third, the
  market could only afford a fixed amount of artists, and when a fake
  creator occupies a position in the pool, some potential new star is
  going to lose his or her chance. In other words, we might lose a great
  artwork merely because someone cheated.

  Some people will not be convinced by the above arguments. They may claim
  that what we need is a law against plagiarism rather than a law of
  copyright. They will argue that if copyright is all about eliminating
  illegal ``borrowing'' the society only needs to check whether an author
  credit all sources appropriately. In other words, the concept of
  copyright should not exist, and everyone can use any intellectual work
  freely. However, in the real world creators do need copyright to provide
  extra protections for their works. We frequently heard stories about how
  authors are hurt by others' abuse of these creations. In \emph{Something
  Borrowed}, Gladwell recounted the story of Dorothy Lewis, who found the
  opera \emph{Frozen} replicating her life story. She felt extremely
  uncomfortable with it because it feels like exposing every secret of her
  to the public. Particularly, ``Lewis'' in the opera had an affair that
  never took place in the real life, which created severe
  misunderstandings between Lewis and her friends (Gladwell 127). One can
  also imagine the following fictional story in a world with only an
  anti-plagiarism law. An artist named Alice created a lovely cartoon
  character. A company used that character in its advertisement without
  explicit permission from Alice. Moreover, that company had a bad
  reputation. People would naturally think that she endorsed that company.
  Not only would Alice feel unjustice---after all, she did nothing
  wrong---but she would also feel helpless as she discovered that there
  was no law protecting her work.

  One might propose to make laws to protect the creators in the above two
  examples. However, it is hard to enumerate all possible accidents that
  can harm authors, thus making a comprehensive law impossible. This
  difficulty is evinced by the fact that the grievance to Dorothy Lewis is
  entirely different from that to the fictional Alice. Instead, if we
  grant the ownership of a piece of intellectual work to its creator, the
  originator could use his or her best knowledge to judge whether it is
  appropriate for someone to use that work. In the case of Dorothy Lewis,
  she could ask the writer of \emph{Frozen} to take her privacy seriously.
  In the fictional example, Alice could directly refuse that company's
  request. Therefore, only a law dedicated to copyright protection could
  make creators free of fears for future harms so that they can be more
  focused on creation itself.

  It is indeed harmful to have a copyright law that grants copyrights of
  abstract ideas, as argued in \emph{RiP!: A Remix Manifesto} (Gaylor
  2008). Nevertheless, the copyright law provides the necessary
  infrastructure upon which our prosperous intellectual market could
  flourish. I believe we can see the hint of how the copyright issue will
  evolve from our experience with the fundamental human rights. Though
  people can voluntarily sacrifice some of their rights for a greater
  good, the protection of human rights can never be given up. Similarly,
  in the future, it might become a \emph{default} strategy for artists to
  license their works to be used freely---maybe with some restrictions
  such as non-commercial use---but creators should always \emph{retain}
  their freedom to decide the fate of their works. This is the future we
  should fight for.

\begin{workscited}
\bibent Widdicombe, Lizzie. ``The Plagiarist's Tale.'' \textit{The New Yorker}.
Feb. 13 \& 20, 2012. Print.

\bibent Gladwell, Malcolm. ``Something Borrowed.'' \textit{The New Yorker}.
22 Nov. 2004. Web. 10 May 2011.

\bibent Tomar, Dave. ``Use Me.'' \textit{The Shadow Scholar: How I Made a Living
  Helping College Kids Cheat}.
New York: Bloomsbury, 2012. 79 -- 87. Print.

\bibent Gaylor, Brett, et al. ``RiP!: A Remix Manifesto.'' Web. 2008.
\end{workscited}
\end{mla}
\end{document}
