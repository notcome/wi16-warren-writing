\documentclass{writing}

\title{2A}
\date{02/03/16}

\begin{document}

\maketitle

In his article \emph{Use Me}, Tomar based his discussion on a single
question: since knowledge is easily accessible from the Internet, is the
traditional role played by professors outdated? He gave an affirmative
answer to this question. First of all, he mentioned the cheating case
happened in Central Florida University, where a professor claimed that
more than a third students were sharing the midterm questions before the
exam (79, 80). However, Tomar pointed out that there was evidence that
students didn't know this was the midterm exam and share it just like
other learning materials (80). He used this to attack these professors
as old-fashioned knowledge ``gatekeepers'' (81), but in order to make
what he argued valid, he should have provided a more direct example
where professors prohibited any information exchange except the one
before a student and the professor. After this, he claimed that the
information age destroyed what, as his friend phrased, ``monopoly on
knowledge'' (81). Though he didn't put it clear, he implied that in the
old times, people needed wealth to acquire knowledge and needed
knowledge to acquire wealth (82). He might be right at this
point---indeed, with the development of the Internet nearly all types of
information are freely accessible to people in most countries---but he
didn't justify this claim rigorously. He then claimed that with
Internet, obtaining knowledge is no longer a process requiring ``the
time, the dedication, {[}and{]} the inquisitiveness'' (Tomar 83). I
didn't see any reason why people can learn writing or category theory
effortlessly with an Internet connection, yet the author didn't justify
it. He also created doubtful images about academia: he said ``academia
hates Wikipedia'' without any evidence, and described universities as if
universities teach for money, which is far from the truth with my life
experience. Again, no supporting evidence.

His argument around intellectual property is also amusing. He gave data
to show that practice breaching intellectual property rights like piracy
was so widespread that it's now a \emph{de facto} standard to gain
information (85) . Yet laws aim at prescribing our behaviors to build a
better society, rather than describing our behaviors like linguists. An
analogy would be that Alice cannot rob Bob and claim that since she is
capable of robbing him so her behavior is legal. At last, he started to
criticize schools again and claimed that the way schools employed the
Internet was an ``illusion of progress''. He tried to express the idea
that schools concentrate knowledge so that they could block people from
learning these knowledge elsewhere. It might be true that schools
concentrate knowledge for the conveniences of their students, but the
latter premise is supported neither by my world knowledge nor the
author's text. Overall, there is hardly any reason to buy what Tomar
said, and if there is one single claim he got right, it's by his luck
rather than reasoning.

The article about Aaron Swartz is more like introducing an exceptional
person rather than debating over the issue of intellectual properties.
The author did make some claims about copyrights, but they were mostly
from the author than from words of Aaron Swartz, so I would like to be
conservative with Aaron Swartz's attitude. Yang clearly intended that
building a paywall to hinder the public's access to ``works of science,
art, and culture'' is wrong (144). From this point of view, all kind of
publishers seeking a profit have moral issues. According to this
principle, the academy is not doing evil, since it did not explicitly
block information flow. Aaron Swartz's own stance, based only on
evidence from the article, is much more puzzling. He started Creative
Commons, developed RSS, built Reddit, stopped SOPA, released
governmental documents, and downloaded papers from JSTOR (Yang 137, 138,
139, 143, 144). At first, it might look like that Aaron should be a
supporter to Internet piracy or someone who thought mashing up is
perfectly fine. However, Aaron fought SOPA not because of its
anti-piracy motivation but because it could allow private companies to
censor the Internet (Yang 139). Creative Commons is a series of
licenses, which is already a protection of intellectual property. It
would be more accurate to say that Aaron supported to free information
that should flow freely. It's hard to determine the definition of
``should'', but clearly academic papers belong to this category.

Neither Aaron nor Gladwell made a very clear principle. It's quite
subjective to say it's immoral and harmful to treat a field as market,
and it's also hard to determine if a work is creative enough. Tomar has
a much sharper stance yet she failed to make a logical justification. I
believe the issue should be discussed on a case-by-case basis. That is,
evaluating different intellectual property protection strategies' impact
on each field. After all, even scientists need money for food.

\end{document}
