\documentclass{writing}

\title{2C}
\date{02/08/16}

\begin{document}

\maketitle

It's hard to define a principle to differentiate plagiarism from
creativity, since new work is developed from old ones rather than
created from nowhere. The more details one considers, the more
complicated the situation is. Therefore, it would be more useful and
approachable to consider the standard in some ``pure'' aspects, namely
separating a work's unoriginality from its usefulness. Here, I would
like to discuss three aspects, what new knowledge does a work create,
does the author deserve what they receive, and is there harm done to the
cited sources' creators.

There are, as I can think of, two dimensions of the creativity of a
work. The first dimension regards the positive effects exerted on the
work's receivers. For instance, an improvement on mathematical notations
merely repackages an old idea, but the new system might inspire other
mathematicians to make substantial progress. The music mashup by
GirlTalk invented no new technique, but it does make a listener feel
high more easily. The effects can be quite versatile even in the same
field. For example, Rowan's \emph{Assassin of Secrets} contributed
neither a new spy novel plot or an interesting writing style (Lizzie
109, 110). However, it is clearly an excellent new work for one who is
purely interested in enjoying yet another spy novel. Initial positive
reactions to the novel confirmed this point (Lizzie 109). The second
dimension on creativity is about making an idea more well-known. The
special relativity proposed by Einstein is creative in this dimension.
But a completely new idea is not always necessary: an English
translation of plato's \emph{Republic}, or a set of notes to
Thucydides's \emph{History of the Peloponnesian War}, makes an idea more
accessible. Though no new idea is added to the pool of human knowledge,
but the average knowledge of every human being is increased, which makes
a new idea more likely to occur. Rowan's novelization of \emph{The
Venturesome Voyage of Captain Voss}, a memoir published in 1913,
rediscovered an old idea that might had be forgotten by the human beings
(Widdicombe 111). From this perspective, it's worth some praise.

Naturally, a question is raised: does Rowan deserve the credit he
received? This question cannot be correctly answered unless we address a
deeper question first: why do we need to reward people when they
contribute something? The first reason for this is to provide incentives
for them to contribute more in the future, as mentioned in the video.
The second reason is that by providing both material interests and
reputations a community implicitly selects someone with good
capabilities. Resources are more concentrated around with those who are
heavily rewarded due to their previous contribution, and these resources
will in turn be used more efficiently. Consider the case of Rowan. By
publishing a popular novel, people expect him to be a \emph{good
writer}, rather than a \emph{good spy novel writer}. People think he
will be capable of writing in general and have high expectations as well
as investments (e.g.~from publishers and editors). These expectations
and investments are destined to be wasted. Moreover, such precious
resources could have been used elsewhere and led to the creation of some
excellent works. Neither did Rowan deserve any kind of incentives. The
society rewards authors because people want to see works of that kind.
Americans have a very high standard of plagiarism when talking about
novels, therefore they felt cheated when they realized that
\emph{Assassin of Secrets} was not that original. A related issue is
honesty. People generally promote honesty as this will increase the
overall efficiency of the society. There is an implicit implication when
Rowan didn't explicitly declared his borrowing: his work was original
according to people's standard to novels. Rowan was aware of this, and
his behavior was cheating. Dishonesty should be punished. \emph{Frozen}
by Lavery, in contrast, was a work that reflects her ability in
composing an opera (Gladwell 127). She did deserve resources
concentrated around her for this particular work. While some might argue
that make an opera as real as possible is also an important ability,
it's not that insignificant. Audience of an opera also didn't have a
presupposition that everything undeclared is conceived by mind or made
from solely the composer's life experience. Even Lavery herself said
that she didn't realize any negative consequence of such borrowing
without permission (Gladwell 126). She didn't cheat on purpose, so there
was no honesty issue.

However, harm was still done. \emph{Frozen} was too similar to what
happened to Dorothy Lewis. As a result, even some fictional part of
\emph{Frozen}'s plot was thought as facts about Lewis's life. This
clearly troubled the original author of the whole story about
investigation into serial killers (Gladwell 127). These negative
consequences are arbitrary and hard to consider in a general way, we
could only discuss them case by case. However, authors of works have a
more clear understanding about potential misuses of their works, either
discovered by themselves or suggested by their friends. Therefore,
author should have freedom to the uses of their works in a reasonable
way. The definition of reasonable ways is the responsibility of laws.
Such freedom should be protected, as otherwise the author might choose
not to publish their works in the first place. From this point of view,
intellectual property is also a property: one can refuse to work for a
property if the rights over that property is not guaranteed. Another
type of harm done to the original authors is much more general. It's
about material interests and fairness of competition. As mentioned
above, we need to protect the idea's owner to encourage everyone's
creativity. It's also about fairness: competitors may choose to copy
someone's work use resources other than that work per se to win the
battle. This not only affects everyone's incentive and make the natural
selection of capable agents less efficient.

\begin{center}\rule{0.5\linewidth}{\linethickness}\end{center}

Why is it so hard to determine whether a creation is original? Because
human knowledge is developed from our previous works rather than created
out of thin air. When some composition's originality is not clear,
whether it is plagiarism or not depends of the reviewer's emphasis on
the creative part or the borrowed part. Therefore, I propose to divide
the standard for judging creativity into three aspects: new ideas
created by the work, whether the author deserves what he or she
receives, and whether there is harm done to the ``lender''. By examining
the issue from these three perspectives, I believe one could respond to
the question of originality in a way that fits best to the circumstance.

\end{document}
