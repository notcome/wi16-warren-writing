\documentclass{writing}

\title{1C}
\date{01/14/16}

\begin{document}

\maketitle

In \emph{They're Watching You at Work}, Don Peck addressed the issue of
applying machine learning based analysis techniques to the field of
estimating one's career potential. The conclusion he arrived was that
big data based ``people analytics'' is good for everyone, in opposite to
the traditional concern that it will deprive many workers' career
opportunity and make the whole market unfair. From those seeking a job
to those holding their positions for many years, from those with high
potential to those less skillful, every member of the society could
benefit from this ``paradigm shift''. He consolidated his claim by
providing large pieces of evidence as well as comparing the new approach
with traditional methods about ``people analytics''.

I find one part of his argument quite troubling and undermining the
credibility of the central claim. He asked a series of questions about
whether we should evaluate one's potential solely based on their
performances in specially designed games, their behaviors reflected in
their browsing history, and their so-called ``data signature''. He
responded by providing several examples of how traditionally people were
hired and explaining that there existed serious bias issue with those
approaches. This, however, is insufficient, because the main claim
requires the premise that the new approach is good in an absolute sense.
Even though delegating the job to machines could relieve the issue of
bias somehow, there are two counterarguments he failed to address.
First, as his interviewees also mentioned, certain characteristics
``learned'' by the machine correlated with ``bias-able'' features, like
one's address and their race. Though companies in this essay did promise
that at least within U.S. those characteristics are not used for
evaluation due to regulation reasons, they did admit such practice
existed in foreign countries. Second, machine learning algorithms have a
common issue that no one can clearly explain how it works. For example,
people don't know the exact meaning of each part of the evaluation
formula given by the machine. This creates a possible bias that no one
could realize. Maybe people with ADHD are judged unqualified
automatically, but everyone---the employer, the employee, and the
evaluation service provider---don't know the ADHD reason and only notice
the bad evaluation results. Both issues are closely connected with the
questions Peck asked, but he seemed to ignore the two, which is
detrimental to the essay overall.

\emph{Visible Man} by Peter Singer discussed a totally different topic.
Instead of writing another criticism on how governments are stealing our
privacy, the author examine the subject on the pros and cons of a more
transparent government. At last, he claimed that making both the public
and the government more open would be a good practice, and its benefit
could outweigh its negative consequences.

To show the main claim, the author had to show both a panoptic public
and a transparent government is overall good. I think he did a good job
at supporting the latter premise. A critical step that he had taken
correctly was to attack directly the counterargument that disclosing
secrets of a government may lead to unexpected consequences. One can
easily find a thousand examples of government hiding political scandals,
but providing only positive examples doesn't eliminate the possibility
that there may be even larger negative consequences. Instead, Singer
provided controversial examples and argued the overall benefits. For
example, he quoted Assange's speech on how the long term benefit of
Kenya violence---helping establish a uncorrupted government, leading
people there to economic prosperity, and ultimately eliminating
unfortunate child death---could surpass the tragedy caused by leaked
documents. He also refuted Clinton's claim on national security by
showing that such secret only consists of a minority of government
affairs. Besides, Singer is honest about the cons of a transparent
government, which makes him more objective and helps readers to make a
calm judgment. Overall, I think he appropriately discussed the issue of
the revelation of government documents and made a very strong point,
which strengthens the main claim significantly.

\end{document}