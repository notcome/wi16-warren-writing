\documentclass{writing}

\title{1E}
\date{01/21/16}

\begin{document}

\maketitle

Have anyone ever had a thought that privacy might be a bad thing? That
might happen when we read a news about how terrorists used Telegram,
another chatting app employing end-to-end encryption, to schedule a
successful attack. But there are still so many downsides about violating
privacy. Nevertheless, in his essay \emph{Visible Man}, Singer made a
bold attempt to make a non-mainstream voice: though people's privacy is
often being violated when using modern technology, the advantages of
such technology outweigh its disadvantages; particularly, people can
harness it to hold the government in check. However, I think not only
did he underestimate the adverse outcome of an open world to ordinary
people, but he also overestimated the benefits brought by a transparent
government. Overall, ``a world without secrets'' creates more problems
than it solves.

When he discussed in what degree government-run large-scale data
collection activity could harm us, Singer presented two unconvincing
arguments. First, he pointed out that there \emph{might} be some
Americans who are proven clean with data collected by the regime. Even
ignoring the fact that there is no datum supporting his argument, we
still could find his imagination ridiculous: it is common sense that
presumption of innocence is a fundamental principle underlying modern
judicial systems; how could the lack of evidence that one is clean be
used to judge they guilty? If this principle had been violated to fight
against terrorism, it is the judicial system which should be fixed
rather than us who should give up our privacy---an idea that even Singer
backed up later in this section. Second, he claimed that an
authoritarian regime can silence opponents with or without electronic
data, a thought which revealed his ignorance about politics and
international relations. On the one hand, we can see that even in
countries as authoritarian as China and Russia, Xi and Putin still need
to employ the judicial systems to put protesters into the jail. On the
other hand, a regime practicing illegal actions will receive strong
oppositions internationally, like Russia's invasion of Ukraine and the
ensuing economic sanctions. In short, in the contemporary international
society, a regime needs justification even for domestics affairs both to
appease its citizens and to avoid objections from other countries.
Besides, online personal information is often the key to reveal the
physical identity of a protester. Therefore, in an authoritarian state,
the more privacy is being leaked, the more danger protesters there are
facing.

Singer may say that his intending audiences are citizens of ``modern''
states, those democratic and free countries. Or he may insist that it's
meaningless to talk about protecting privacy in countries like China.
Let's assume we do live in an ideal society where the government do not
use data collected against its citizens---namely we have made ``the
crucial step'' he mentioned. Still, one could notice that he completely
neglected the issue that corporations do evil with users' information.
This is actually a topic more controversial than that of a government's
abuse of personal information. It's a well-known practice that companies
like Google collecting users' information and selling advertisements for
profit. From a certain point of view, these companies which claim that
they provide free products are in fact selling their customers to other
enterprises. There are two major problems with this practice. First, as
Singer also mentioned in his essay, such a large-scale activity
inevitably induce some people to do evil. Unlike the government case, we
can hardly make a company open or democratic. Second, ``selling
customers'' is so profitable that in many fields this is the only way to
be successful, and in the long term---after other types of companies
die, the customers have no choice other than giving up privacy. While
some might say that they don't care their privacy and would prefer to
give up theirs for better services, we should realize how this situation
is similar to a monopoly.

Another issue Singer neglected is that neither the government nor
corporations need to play an active role to make large-scale information
gathering harmful to a society. As most people know, Apple is now at the
center of the debate about whether the government should implant
backdoors in electronic devices. The argument is that it's impossible to
have a backdoor accessible to only one party; either there is a backdoor
that may be exploited by anyone, or there is no security compromise at
all. It's likely that crackers---those who employ computer technology
for illegal purposes, traditionally known as hackers---managed to obtain
personal information from the government so that they can set up spams.
It's also unconceivable that ISIL's followers acquire physical locations
of those actively denounce terrorism. Then, violation of privacy is no
longer a moral issue; it's about real money and real lives. While some
may say our data are quite safe on these servers, we should not forget
that even documents stored in the American government can be stolen by
WikiLeak, not to mention those Silicon Valley startups.

\begin{center}\rule{0.5\linewidth}{\linethickness}\end{center}

Like when proving a set is identical to another rigorously we need to
prove each set is a subset of the other, when weighing the pros and cons
of ''a world without secrets'' we should also examine each side
carefully. Since I have demonstrated that Singer underestimated the
adverse outcomes of giving up the privacy of the general public, now
it's time to see his argument about how a transparent government could
be helpful. Unexpectedly, his discussion of this section is also full of
logical errors and shows his ignorance of common sense.

Singer claimed that a transparent government is beneficial and gave the
example of WikiLeak. I do agree with him on this very point, but I don't
think the benefits of transparency overweigh the price we need to pay
when we give up our privacy. Instead, I think he also overestimated the
power of openness. He quoted Assange's belief to express his idea that
people need more information to make accurate judgements. It is true
that information is necessary to make accurate judgements, but this does
not imply that with sufficient information people will make judgments.
Why? Most people are to lazy to absorb the knowledge of the world. This
is clearly reflected from the fact that even in 2016 when everyone is
talking about human rights or freedom, Donald Trump could still easily
use words that can not even convince himself to attract more than ten
million supporters in the world's most developed country. Besides,
knowing the issue does not mean that we can solve them. Even without
open data every college students could realize that their textbooks are
overpriced, but is there any action taken against it? Well, there is,
and Aaron Swartz committed his suicide. This is a problem known as
collective action problem. The price to solve a problem for an
individual is so high that everyone wants to free ride. A mass amount of
people can hardly achieve anything: they may be willing to sign a
signature, but much less willing to demonstrate, not to mention spend a
significant amount of money to advertise their demands. Some may say
with enough attention NGOs will come from nowhere and lead the people,
but I would like to say that the very environment in which NGOs form is
not a transparent government but a rule of law society with sufficient
institutions that protect individual rights. There is indeed something
very beneficial about a transparent government---the panoptic effect.
The idea is that the government officer do not need to be supervised by
a dedicated group of people, because as long as the officer knows there
is a risk in doing illegal things, they would not choose to do it.
Though this perfectly solved the problem of collective action I
mentioned above, when the interest is too significant, I believe any
politicians are willing to risk their careers, just like those who
committ crimes even though they know the laws. We still need a series of
institutions specialized in keeping the government incorruptible.
Unfortunately, Singer, though mentioning the panopticon to make his
essay more fancy, didn't consider this point at all; and as I have said,
this model does bring benefits but not large enough.

Openness do increase the costs of governments' misconducts and eliminate
myriad ordinarily evil ideas. However, why do both the people and the
government have to accept transparency? Singer could simply argue that a
transparent government is more beneficial---it is, though its benefit
cannot pay the price of asking everyone to give up their privacy. Even
the author himself might have realized the overall weakness of his essay
and therefore chose to discuss the benefits of people's accepting
openness of themselves in the last section. However, are these examples
what we privacy defenders fight against? Certainly not. It's natural
that one should give up some freedom and obey rules in a public space,
and it's reasonable therefore to employ certain strategies to increase
obedience---as long as our rights in our private space are not violated.
The second and the third examples are even more ridiculous: how could
this count as being observed? In the donation example, we can safely say
that callers are merely affected by others' altruistic behaviors instead
of the notion that they are observed. In the energy consumption example,
we can also say that these customers merely want to save their money.
The author do not fully address these possible counterarguments. In
contrast, to see how terrible a panoptic society would be, one can
simply pick up a history book of Nazi German or schedule a visit to
North Korea.

In conclusion, Singer failed to analyze the negative impact of violating
privacy systematically, from authoritarian governments, profit-seeking
Internet companies, to crackers as well as terrorists. In addition,
though his claim that a transparent government has advantages is
correct, not only did he miss an important reason why this is the case,
his argument on how people can make better judgements do not necessarily
hold. It is laws and institutions which are playing a much more vital
role in ensuring our society works as desired. Last but not least, there
is no logical relation between abandoning citizens' privacy and making
our governments transparent. He could have written a better article if
he did not intend to justify these anti-privacy conducts. Unfortunately,
he chose the road less traveled.

\end{document}