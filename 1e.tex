\documentclass{writing}

\title{1E}
\date{01/21/16}

\begin{document}

\maketitle

Have anyone ever had a thought that privacy might be a bad thing? That
might happen when we read a news about how terrorists used Telegram,
another chatting app employing end-to-end encryption, to schedule a
successful attack. But there are still so many downsides about violating
privacy. Nevertheless, in his essay \emph{Visible Man}, Singer made a
bold attempt to make a non-mainstream voice: though people's privacy is
often being violated when using modern technology, the advantages of
such technology outweigh its disadvantages; particularly, people can
harness it to hold the government in check. However, I think not only
did he underestimate the adverse outcome of an open world to ordinary
people, but he also overestimated the benefits brought by a transparent
government. Overall, ``a world without secrets'' creates more problems
than it solves.

Singer presented two reasons why government-rule large-scale data
collection activities have less adverse outcomes than people usually
thought, but neither of them are convincing. First, Singer pointed out
was that there \emph{might} be some Americans proven clean with data
collected by the government. Even ignoring the fact that there is no
datum supporting his argument, we could still find his imagination
ridiculous. It is common sense that \emph{presumption of innocence} is a
fundamental principle underlying modern judicial systems, which says
that the accused don't need to provide evidence to convince they are
clean---it's the prosecutors who should prove they guilty. Therefore,
the existence of collected data that prove citizens clean, as Singer
suggested, would be unnecessary. One may argue that under certain secret
cases the government might have violated this principle when judging
American citizens. However, in such case it's the judicial system which
should be fixed; the benefit brought by giving up privacy is \emph{ad
hoc} and should not be used to justify such practice. The second
justification Singer gave was that an authoritarian regime could silence
opponents, ``with or without electronic data''. The phrase can be
interpreted in two ways: either he meant an authoritarian regime could
silence its opponents with only non-electronic data, or without any form
of data. From the context, I believe he meant the second interpretation.
Such a claim are clearly not in accord with the real world. On the one
hand, even in countries as authoritarian as China and Russia, Xi and
Putin still need to employ the judicial systems to put protesters into
the jail. To maintain steady development a society requires order, and
eliminating opponents arbitrarily can create terror and destroy such
order. As a result, most authoritarian leaders need to \emph{rule by
law}; even though they can ``maximize'' evidences against their
opponents, they still need some piece of evidence to start with. On the
other hand, online personal information is often the key to reveal the
physical identity of a protester. Therefore, in an authoritarian state,
the more privacy is being leaked, the higher cost protesters are
spending hiding themselves. This reason also weakens Singer's claim if
what he intended was the first interpretation discussed above. To sum
up, exposing privacy to the governments is more harmful than Singer had
analyzed, which weakens his overall claim.

Leaking privacy to corporations or criminal groups could also be
detrimental, which are ignored by Singer nearly completely. In fact, to
most people data collection by these organizations are more troublesome
than those conducted by the government. First, data collection by
corporations is a universal practice; as Singer mentioned in his essay,
such a large-scale activity inevitably induce some people to do evil.
What makes things worse is that unlike the government, except for
certain unusual cases, we cannot make a company open and that without
direct supervision big corporations are even more likely to do evil.
Second, the Internet is a dangerous place where crackers---those who
employ computer technology for illegal purposes, traditionally known as
hackers---hunt around. It's likely that these criminal groups manage to
steal personal information from the government or Internet companies and
utilize the information to setup spam. It's also conceivable that
terrorists like ISIL's followers acquire physical locations of those
peace fighters. From this perspective, disclosure of privacy might
threaten even the most ordinary citizens' wealth and lives. While some
may say our data are quite safe on servers, we should not forget that
even documents stored in the American government can be stolen by
WikiLeak, not to mention data stored at those Silicon Valley startups.
Since the author only mentioned that big corporations might be an issue
when describing the open nature of our society and he ignored criminal
groups completely, his claim that giving up privacy is not as bad as
people often think is further undermined.

\begin{center}\rule{0.5\linewidth}{\linethickness}\end{center}

After refuting Singer's premise that the negative impact of a ``visible
world'' is more trivial than most people thought, it's time to analyze
why the positive outcome of this kind of openness is not as much as
Singer promised. Besides, I think the benefits brought by openness is
actually a direct consequence of good institutions in different aspects
of the society rather than this particular kind of institution. In
Singer's imagination, an ideal society should be a place where some
organizations committed to exposing the government's misconducts to the
general public in order to restrain the government. He also quoted
Assange's belief to express his idea that people need more information
to make accurate judgements. However, though it is necessary to have
accurate information to make good judgements, the availability of
accurate information doesn't guarantee right judgements. People are too
lazy to absorb the knowledge around themselves. How many people will
read WikiLeak's documents by themselves? Because we have a whole
occupation dedicated to reading documents for others---we call them
journalists---it's reasonable to infer that the number is small.
Consequently, there must exist a myriad of blind spots about the
government that the general public will never know, even if the
information is \emph{available}. This creates a vicious cycle: a hot
topic will draw the attention of so many people that even if there is no
one actively digging information about it, someone is going to leak this
information; a lost topic, though worth the spotlight, will not be
mentioned by the media and thus disappear from everyone. It is because
of this feature of communication that the benefit of a transparent
government is severely hampered. Moreover, knowing the issue does not
mean that we can solve them. Even without open data every college
students could realize that their textbooks are overpriced, but is there
any action taken against it? Well, there is, and Aaron Swartz committed
his suicide. People need NGOs to gather their strengths for a common
objective. The question whether the political sphere makes NGOs'
development obstacle-free remains debate, yet it is well known that a
rule of law society with sufficient institutions that protect individual
rights is necessary for such organizations to grow significant. From
this perspective it is clear that a transparent government solely is
meaningless---we need other institutions. There is indeed something very
beneficial about a transparent government---the panoptic effect---but
unfortunately it is not discussed by Singer. The idea is similar to the
panopticon: a government does not its every employee to be supervised by
some dedicated group at all times, since as long as the officer knows
that their behavior at some time might be examined, they won't risk
their career to do illegal things. However, to make the model work as
expected, it is necessary to isolate government officers from their
supervisors, or they might seek rent together. The general public's
interest has little intersection with that of officers who are
interested in rent-seeking. Therefore, they are indeed good supervisors.
Still, this requires a combination of institutions to protect people's
freedom. All in all, since the public is inefficient in knowing the
government and working together, the benefits of a transparent
government are not as significant as Singer described; though the
panoptic effect will appear with transparency, this is a topic ignored
by the author altogether; most importantly, all of these benefits are
built upon a series of institutions, the efficiency of which is also not
discussed by him.

Perhaps the most ``outstanding'' logical error of this essay is that
Singer binds the openness of the public with the transparency of the
government. Is this necessary? No. In fact the Americans, just like
citizens in most well developed countries, already have a quite
transparent and honest government. Singer could simply argue that a
transparent government is more beneficial---it is, though its benefit
cannot pay the price of asking everyone to give up their privacy. Even
the author himself might have realized this problem and therefore chose
to discuss the benefits of people's accepting openness of themselves in
the last section, but his argument's validity is again questionable. The
first example he gave was that people behaved more honestly in the
public if they were told to be watched. Yet the privacy issue people
concern is their privacy in their private life---it's natural to give up
some freedom and obey rules in a public space, and it's reasonable
therefore to employ certain strategies to increase obedience. In the
third example, when an energy company presented its customers with both
their own energy use and the average use in that area, the overall
energy consumption reduced. Despite the fact that personal information
is anonymized, it's likely that users merely want to save their
expenses. The author do not fully address these possible
counterarguments. In contrast, to see how terrible such an open society
would be, one can simply pick up a history book of Nazi German or
schedule a visit to North Korea.

In conclusion, Singer failed to analyze the negative impact of violating
privacy systematically, from authoritarian governments, profit-seeking
Internet companies, to crackers as well as terrorists. In addition,
though his claim that a transparent government has advantages is
correct, not only did he miss an important reason why this is the case,
his argument on how people can make better judgements do not necessarily
hold. It is laws and institutions which are playing a much more vital
role in ensuring our society works as desired. Last but not least, there
is no logical relation between abandoning citizens' privacy and making
our governments transparent. He could have written a better article if
he did not intend to justify these anti-privacy conducts. Unfortunately,
he chose the road less traveled.

\end{document}
