\documentclass{writing}

\title{Syllabus Quiz}
\date{01/06/16}

\begin{document}

\maketitle

This program builds on the idea that “learning is an evolving, participative, and collaborative endeavor”. In addition to its high demanding on each student, a significant portion of this course takes the form of collaboration—discussions, peer reviews, etc. From these principles, we can summarize the course’s policies rationally.

To begin with, a student’s commitment to conferences are not only essential to their learning but will also affect the collaboration between everyone in the course. Therefore, each significant failure of participating in a conference, which will be defined below, is counted as an “absence”, and more than two absences may lead to failing the class and advised withdraw. First, attendance is mandatory and missing a conference is an absence. Second, since coming late to the class—though having a less negative impact on an individual—will still disturb other students and hamper the discussion, being late for more than five minutes more than twice will count as an absence. Last but not least, because using of electronic devices, especially phones, is also distracting to everyone in the class, persistent use of such technology is also an absence.

Similarly, failing to submit drafts and revisions on time is also detrimental to every student in the class and will be severely penalized, because such incident is more than a reflection of the lack of commitment from the student in question: it also slow down the whole course’s progress. An exception is to make a special agreement with the instructor well before the paper’s deadline. When a late paper is inevitable, one should manage to reach an agreement with the instructor on when and how the paper will be turned in. In such case, there will be grade penalties at the disposition of the instructor, with an upper bound of lowering one-third of the grade for each graded assignment. The penalty may be increased for each day after the paper is due. Since the excuse of a late paper will affect one’s grade, falsifying or fabricating the fact is a violation of academic integrity.

Lastly, it seems fair to adjust one’s grade based on one’s commitment—as different students start with different writing backgrounds. The factors to be considered are presence and participation, peer work, reflection, and overall improvement. The adjustment process, known as process grade, which takes place after grading each of the three final essays, can raise or lower one’s grade by one-third of a letter but may also put the grade intact. Process grade is determined by the instructor.

\bigskip
\hrule
\bigskip

I was in an ordinary Chinese high school which provided no English writing education—unless you call a three-paragraph 120-word piece of text an essay. I did receive some amount of training in English grammar in school and practiced some SAT/ACT grammar questions. There were some though sparse training on how to write a critical essay in Chinese, but my performance in those lectures was not so good.

\bigskip
\hrule
\bigskip

I find the second topic very intriguing to me. I have a lot to say on piracy and plagiarism. The fact that several industries have been severely damaged due to disrespect to intellectual properties touched me. I don’t have any comment or concerns on this course.

\end{document}
