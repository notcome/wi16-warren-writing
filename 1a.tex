\documentclass{writing}

\title{1A}
\date{01/07/16}

\begin{document}

\maketitle

It was two weeks before the final week of the last quarter. My friend
and I managed to spare some free time to watch a movie at UTC. After an
initial search, I narrowed down the movie list to three items, \emph{The
Hunger Games}, \emph{The Martian}, and Pixar's \emph{The Good Dinosaur}.
I was not a fan of \emph{The Hunger Games} but had accidentally watched
the part I of Mockingly---I hoped to finish what I had started.
\emph{The Martian} is a science fiction about astronauts on Mars, with
no supernatural abilities, which was the taste I liked. For \emph{The
Good Dinosaur}, I was such a fan of Pixar that I unconditionally trust
movies made by it.

Picking the right movie is not something as insignificant as one might
think. I often hear people say that what matters is the experience with
friends rather than the movie per se, but I disagree: choosing the wrong
movie will not only lead to a disastrous experience but also lower my
image in my friends' mind. However, the decision making is pretty
complex, partially caused by some inherent nature of movies. No one
wants a spoiler, yet a detailed description of the plot is essential to
an accurate estimation of a movie's quality. An audience can only have a
basic familiarity with the setting of a movie, from trailers or movie
posters. In this case, all three movies' backgrounds seemed fine to me.
The other characteristic that can determine a movie's quality is its
performance. One can ask their friends who have already watched that
movie. Downsides of this approach include that such comment is entirely
subjective and that the small size of samples is often insufficient to
eliminate emotional bias.

Nevertheless, asking friends was the only thing I could do at that time,
so I did it. Several friends consistently told me that \emph{The Hunger
Games} was a poorly made movie. I could cross it off my list. None of my
friends had watched the rest two movies, but with the rapid development
of the Internet, I could conveniently access thousands of reviews and
ratings the first day after the movie's release. A Google search showed
that both movies were of good quality, although \emph{The Martian} had a
slightly higher approval rate. That night, I watched \emph{The Martian}
with my friend.

Still, the new technology is neither perfect nor cost-free. The problem
of spoiler is not resolved by gathering reviews and ratings
together---one may still risk knowing that Han Solo is died by reading
any in-depth reviews on Star Wars. In fact, however the technology
advances, I don't think one can judge both the plot consciously and keep
a fresh feeling of that movie. Another more fundamental issue of online
reviews, however, is that personal bias still exists even within a
relatively large amount of samples. Here is a personal example. While
the whole Internet seems to love Star Wars VII, I found it simple plot
boring. Unlike a lot of people, I love the complicated political stories
told in the Star Wars prequel. In other words, what matters is not the
general attitude to a movie but the opinions of those who share similar
tastes with me on movies.

Fortunately, new technology does solve this problem in related
fields---recommending TV series, movies that have been released for a
while, and music. While I could not use such technology to compare
movies in theaters, I can always get some pretty good recommendations on
TV series whenever I have time to watch a new one. The basic idea of
this approach is that if one share the love on several things with
another person, then if one of them find some new thing interesting, the
other one is also likely to have a taste for that thing. This technique
is quite convenient, as all one need to do is to tell the machine
several things they like and then enjoy the accurate recommendation!
Nevertheless, nothing is flawless. The first and foremost concern is
privacy---service providers will know a lot about their customers, and
they have the capability to abuse these data. Another issue is that
machine can only give recommendations based on previous data; they
cannot recommend movies that are not yet released.

With the advance of technology, we can go to theaters much more
confidently. The application of machine-based recommendation to newly
released movies is on the verge of emergence, and researchers are
working hard to pushing machines to new extremes. Hopefully, one year
later even Siri could tell me which movie to watch.

\end{document}
