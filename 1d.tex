\documentclass{writing}

\title{1D}
\date{01/19/16}

\begin{document}

\maketitle

In his essay \emph{Visible Man}, Singer presented an interesting claim: though people’s privacy is often being violated when using modern technology, the advantages of such technology outweigh its disadvantages; particularly, people can harness it to hold the government in check. He discussed at length the example of WikiLeak. This organization has disclosed shocking information about governments all over the world several times, and its behaviors did bring some good to this planet. He consolidated his claim by showing the government can do evil with or without citizens’ data, by refuting the counterargument that transparency in government may threaten national security, and by giving experiments that show people behave more justly under other’s inspection.

However, I think not only did he attack a straw man argument when assessing the adverse outcome of an open world to ordinary people, but he also overestimated the benefits brought by a transparent government. On the one hand, the government is not the only one interested in people’s privacy, so the violation of privacy has a far greater negative impact than he has estimated. On the other hand, the nature of the general public hinders their ability to effectively fight against misconducts of the government. Overall, “a world without secrets” creates more problems than it solves.

\end{document}