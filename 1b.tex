\documentclass{writing}

\title{1B}
\date{01/12/16}

\begin{document}

\maketitle

The excerpt from \emph{Always On} has one central claim: with the
advances of Internet, people disconnect themselves from the real world
and turn to a virtual life interacting with human beings inhumanly.

The first section regards the premise that it's an
inevitable ongoing evolution that people are getting further away from
their traditional face-to-face social life. Turkle first provided direct
evidence to support this: for example, even during his travel to Paris
with his daughter, the latter told the author that she didn't feel
disconnected with her friends in Boston. Then, he argued that the
Internet is a convenient solution to Americans' feeling of insecure,
isolated, and lonely, a fact that is supported by well-known research
results.

The second section continues the discussion on the public psychological
problems and claims that people are becoming accustomed to a life
parallel to the real world, and that the boundaries between
traditionally isolated environments are getting blurred. Turkle started
with a syllogism: we human beings are always interested in ``working on
identity''; it's easier to create a perfect yet false image of ourselves
online than doing offline; therefore, we invest more and more online.
Subsequently, he used a story from a \emph{Second Life user} to show
that mobile technology dramatically simplifies the transition between
real life and virtual life, leading to the so-called \emph{life-mix}.
From this point, supplied with examples from \emph{life-mix} critics,
the author arrives that technologies have transformed our world into a
``continual partial attention'', which blurs the boundary between work
and life.

In the last section, the author employed a natural transition from the
\emph{life-mix} to multitasking. Using evidence including his own
teaching experiences, Turkle showed that multitasking feels good. In
pursuing such satisfaction, however, people quickly found themselves
overwhelmed by the burden information. Several examples are given,
including one from a curator at a museum. Hence, Turkle concludes that
as transformed from an identity-seeker into a machine specializing in
handling the information stream, everyone is turned by the new
technology from an old-fashioned social animal into a machine
indifferent to other ``robots'' in the online community.

The second essay is on the question of using the Internet's side effects
to human brains. It uses a hierarchy structure to argue the claims that
the new technology gives us access to a vast amount of information at
the cost of our ability in thinking deep.

The first sub-claim is that the Internet is adept at catching and
wasting our attention. Carr talked at length about the Internet's
ability to make users stay, from the diversity of information, the
instant response and rewards, to the high interactivity of this
technology. However, by doing so---creating lots of distractions---the
Internet successfully prevents us from thinking deep. A counterargument
is that distractions are useful under certain situations, but the author
showed that prerequisites for such unconscious thought are not met when
browsing the Internet.

Next, Carr presents several experiments on how using Internet could
change our mind in a very short period. He then gave an explanation for
such evolution: several types of shallow thinking, like choosing the
links to browse, are required and repeatedly trained when using the
Internet. A counterargument is given, which says that such stimulation
of brain might be a good thing. The author refuted it reasonably,
started a discussion on the mechanism of our brain, and explained why
the Internet makes us a mindless information consumer rather than
feeding us with useful knowledge.

Following this discussion is three pieces of evidence on the Internet's
inefficiencies on teaching us. Carr presented a set of experiments to
show that neither hypertext nor hypermedia is helpful in learning, both
of which are abundant online. He also considered the fact that carefully
designed multimedia materials are indeed helpful, but he argued that
such resources are hardly available on this fast-paced web. At last, he
discussed how Internet users, including researchers, changed from
reading to power-browsing, by providing evidence that people read web
pages less carefully than traditional books---in fact, they merely skim.
Up to this point, Carr has rigorously proved his claim that while the
Internet is supplying an insane amount of information, it doesn't help
us think in a serious way.

\end{document}