\documentclass[12pt,letterpaper]{article}

\usepackage{ifpdf}
\usepackage{mla}

\begin{document}
\begin{mla}{Minsheng}{Liu}{Marissa Perino}{WCWP 10A}{02/18/16}{2G Reflection}

There are some skills I should have developed with my last paper.
However, since I chose to criticize the article by Singer in my first
essay, I failed to practice them.

First and foremost, I did not need to actively find a level three
question. All I needed to do was to stick with Singer's arguments. This
time I was not so fortunate---though I could have done the same thing,
by criticizing Tomar; I chose a harder prompt on purpose---so I made a
poor introduction with my discovery draft. At that time, what I was
\emph{writing} was that we should analyze the issue of plagiarism from
three different aspects. There is nothing arguable here. Luckily, by
putting my proposed principle in the big context, I figured out that I
could simply go against copyleft, which is a level three question.

Second, contextualization. As I described in the last paragraph, I found
my main claim by looking at my idea in context. This is something missed
from my first paper. I have to admit that I didn't use sources other
than the Singer's article in a meaningful way. Except for the nature of
an article review, I didn't \emph{engage} into the conversation. My
interaction with those writers is most clearly reflected by the fact
that I picked my content based on points made in these sources. For
example, Gladwell expressed his concern on harms done to authors
(Gladwell 127), and that's the theme of my third sub-claim.

I believe the finding the motivating question is the hardest aspect.
After I have found the question and evaluated it in the context, the
article came naturally. There was no other way to organize the whole
article. However, a significant amount of work was required to make my
article readable. I developed two strategies to solve this issue. The
first is to let my peers read my paper. They pointed out places where my
analysis is not clear enough as well as places where I put something
implicit. It is difficult for me to keep track of every single step of
reasoning in my mind. Moreover, some ideas are so familiar to me that I
take them for granted. Therefore, I believe small conferences are
crucial and this course should enforce it after every nontrivial
project.

The second strategy is to let myself re-read my own paper a few
hours/days after I finished my draft. As the instructor pointed out, I
needed to ``anticipate what a common reader'' would think. Re-reading
after some time can make my mind fresh so I can notice things like
strange order of sentences. An observation is that my first draft shares
little paragraph-level structural similarities with my latest version. I
would ask myself to explicitly rewrite the paper as a way to improve
readability.

\subsubsection*{Works Cited}
\bibent Gladwell, Malcolm. ``Something Borrowed.'' \textit{The New Yorker}.
22 Nov. 2004. Web. 10 May 2011.
\end{mla}
\end{document}
