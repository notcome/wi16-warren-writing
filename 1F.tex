\documentclass{writing}

\title{1F}
\date{01/27/16}

\begin{document}

\maketitle

Has anyone ever had a thought that privacy might be a bad thing? That
might happen when we read a news about how terrorists successfully
scheduled an attack using Telegram, a chatting app employing end-to-end
encryption. In his essay \emph{Visible Man}, Singer made a bold attempt
to make a non-mainstream voice: though people's privacy is often
violated when using modern technology, the advantages of such technology
outweigh its disadvantages; particularly, people can harness it to hold
the government in check. However, I think not only did he underestimate
the adverse outcome of no privacy protection, but he also overestimated
the benefits brought by a transparent government. Overall, ``a world
without secrets'' creates more problems than it solves.

Singer presented two reasons why government-rule large-scale data
collection activities have less adverse outcomes than people usually
thought, but neither of them is convincing. First, Singer pointed out
was that there \emph{might} be some American suspects proven clean with
data gathered by the government (34). This justification is untenable
even if we ignore his lack of evidence. Underlying modern judicial
systems is a fundamental principle called \emph{presumption of
innocence}: the accused don't need to provide evidence to convince they
are clean; it's the prosecutors who should prove them guilty. Therefore,
it's needless to show a citizen clean using collected data. One may
argue that sometimes the government might have violated this principle.
However, in such case we shall fix the judicial system (Singer 34). The
benefit brought by breaching privacy is \emph{ad hoc} and cannot justify
such practice. In his second argument, Singer said that an authoritarian
regime could suppress opponents, ``with or without electronic data''
(34). The quoted phrase has two interpretations: either he meant
authoritarian rule could silence dissidents with only non-electronic
data, or without any form of data. From the context, I believe he meant
the second interpretation. Such a claim is clearly not in accord with
the real world. Even in countries as authoritarian as China and Russia,
Xi and Putin still employ the judicial systems to arrest protesters. A
society requires order for steady development, but arbitrarily
eliminating opponents can destroy it. As a result, most authoritarian
leaders need to \emph{rule by law}. Even though they can ``maximize''
evidence against their opponents, but they cannot create evidence from
nowhere. Moreover, online personal information can reveal the physical
identity of a protester. Therefore, in an authoritarian state, the less
privacy is protected, the higher cost protesters are spending hiding
themselves. This reason also weakens Singer's claim if what he intended
was the first interpretation discussed above. To sum up, exposing
privacy to the governments is more harmful than Singer had analyzed,
which weakens his main claim.

Leaking privacy to corporations or criminal groups could also be
detrimental, but Singer completely ignored it. To most people,
data-gathering policies by these organizations are more worrisome than
those conducted by the government. First, it's a universal practice for
corporations to collect customers' information (Peck 10). As Singer
mentioned in his essay, any large-scale activity inevitably induces some
people to do evil (33; 35). Therefore, companies that collect data are
unavoidably unsafe. What makes things worse is that we cannot enforce
open, which gives it more chances to do evil. Moreover, crimes are
prevalent on the Internet. It's likely for a hacker to steal personal
information from the government or Internet companies and utilize the
information to set up spam. It's also conceivable that terrorists can
acquire physical locations of those peace defenders. From this
perspective, the disclosure of privacy might threaten the most ordinary
citizens' wealth and lives. While some may say our data are safe on
servers, we should not forget that even documents stored in the American
government can be stolen by WikiLeak, not to mention data stored at
those Silicon Valley startups (Singer 32). Since the author only
mentioned that big corporations might be an issue when describing the
open nature of our society (Singer 33) and he ignored online crimes
completely, his attempt to justify forgoing privacy is further
undermined.

Though Singer was right that a transparent government is good (34), the
benefit of this openness is not as much as he had analyzed. To Singer,
in an ideal society, there should be some organizations exposing the
government's misconducts to the general public so that the latter will
restrain the government (34). He also quoted Assange's belief that
people need more information to make accurate judgments (Singer 36).
However, though it is necessary to have information to make sound
judgments, the availability of information doesn't guarantee right
decisions. People are too lazy to discover facts about the world. For
instance, it's reasonable to believe that not so many people have read
WikiLeak's documents by themselves because we have an occupation known
as journalists dedicated to reading materials for others. Consequently,
there must exist numerous blind spots about the government that the
general public will never know, even if the information is
\emph{available}. A hot topic will draw the attention of so many people
that even if there is no one actively digging information about it,
someone is going to leak something; a lost topic, though worth the
spotlight, will not be mentioned by the media and thus be forgotten
forever. This characteristic of communication hampers the benefit of a
transparent government severely. Moreover, knowing an issue does not
mean that it can be solved. Even without open data, every college
students could realize that their textbooks are overpriced, but is there
any action taken against it? Well, there is, and Aaron Swartz committed
his suicide. People need NGOs to unite their strengths for a common
objective. It remains debate whether our political sphere makes the
development of NGOs obstacle-free, which raises a new doubt about
Singer's claim.

An undeniable advantage of a transparent government is the
\emph{panoptic} effect. Unfortunately, Singer talked Panopticon merely
to ``fancify'' his article and didn't discuss it in-depth (31; 36). The
idea is simple: the government does not need its every employee to be
always supervised by someone. As long as officers realize that someone
might be examining their behaviors, they won't risk their careers to do
illegal things. To make the model work, it is necessary to isolate
government officers from their supervisors, or they might corrupt
together. Because the general public's interest has no intersection with
that of civil servants interested in rent-seeking, people are good
overseers. All in all, the public is inefficient in both knowing the
government---regardless of the latter's openness---and fighting
injustice. The panoptic effect will appear with transparency, but Singer
didn't discuss this topic at all. Therefore, the author failed to show
the advantages of a transparent are significant enough.

At the last section, Singer returned to the topic of why privacy is not
that important, yet he failed again. This time, he tried to show that
people behaved better when their privacy was deprived. The first example
he gave was that people were more honestly if they were told to be
watched. However, the privacy issue people concern is their privacy in
their private life. It's natural to give up some freedom and obey rules
in a public space, and it's reasonable therefore to employ certain
strategies to increase obedience. In the third example, when an energy
company presented its customers with both their energy use and the
average use in that area, the overall energy consumption reduced. Not
only was the personal information anonymized, but the author also failed
to consider the counterargument that users merely want to save their
expenses. Overall, his analysis is weak. To see how terrible a society
deprived of privacy right would be, one can simply pick up a history
book of Nazi German or schedule a visit to North Korea. (Singer 36)

I believe the reason Singer returned to the topic is that he discovered
the biggest flaw implied in his central claim: to obtain a transparent
government the public must renounce their rights to privacy. Is this
necessary? No. In fact, the Americans, like citizens in most
well-developed countries, have a transparent and honest government while
maintaining their privacy. He failed to analyze the negative impact of
violating privacy systematically, from authoritarian governments,
profit-seeking Internet companies, to criminals as well as terrorists.
Though his claim that a transparent government has advantages is
correct, he could hardly convince readers to believe that these benefits
outweigh the price of depriving everyone's privacy. Singer could have
composed a strong argument if he did not intend to justify activities
that breach people's privacy. Unfortunately, he chose the road less
traveled.

\end{document}
