\documentclass{writing}

\title{2D}
\date{02/10/16}

\begin{document}

\maketitle

Has intellectual property become a concept of the past? Indeed, in the
age of Internet, sharing without permission has become a \emph{de facto}
standard, as described by Tomar in his book \emph{Use Me} (85).
Moreover, other people are arguing that the idea of copyright is wrong
in the first place, because new knowledge is always created on the basis
of the old one. In his article \emph{Something Borrowed}, Malcolm
Gladwell gently supported the practice of using ``old words'' to
establish ``new ideas'' (127). In \emph{RiP!: A Remix Manifesto}, Brett
Gaylor criticized the copyright laws, showing how the current copyright
protection institution impeded artists' creativity (Gaylor 2008). Though
I partially agree with the view that the domain of copyright should be
limited---more specifically, I disagree with patenting ideas---I insist
that the protection of intellectual property is necessary for soliciting
creativity. Not only does it promote fairness among creators, it also
protects them from unintended harm. Overall, copyright is more
beneficial than copyleft when done right.

The domain of intellectual property should be the commitment part of
creative works, rather than the part of idea---by idea, I refer to
something general and simple, for example, a concept. There are mainly
two reasons. First, it's hard to find out the author of a particular
idea. For example, if Google intends to patents its design language
\emph{Material Design}, it should be the metric specification and its
choice of color platte that gets protected. The concept of
``card''---that a group of information should be put in a rectangle with
3D border visual effects---has already been used in different places for
many years. Even though Google formally name this idea, it does not own
the idea. Looking into the history, we could find that a similar idea
was already embodied in bulletin boards, an invention before graphical
user interface. Idea is also hard to attribute in literature. In
\emph{Something Borrowed}, Gladwell mentioned that the opera writer
Lavery borrowed his phrase ``sins and symptoms'', yet the phrase was a
quote by Gandhi, who probably borrowed it from somewhere else (128).
It's clearly unreasonable that when we don't know the real author of an
idea, we grant that idea to someone simply because that person requests
the ownership first. The other reason why intellectual property should
not be applicable to ideas is that ideas are scarce. For example, there
are only a few possible plots in spy novels (Widdicombe 52). If they are
patented, every author of spy novels will likely be a plagiarist.
Besides, protecting general ideas can easily lead to monopoly. For
example, the first company of a new type of business can simply declare
this business model as their intellectual property and charge every
followers a royalty. Moreover, greedy people will be attracted: they
will try their best to gain ownership of every general idea that is
already widely used in the real life, since if they succeed, they can
gain an insane amount of royalties. Therefore, granting ownership of an
idea is detrimental to the creativity of our society. Copyright law
shouldn't be done this way. After all, as mentioned in \emph{RiP!: A
Remix Manifesto}, the very principle behind the first copyright law was
to give incentive to the creator (Gaylor 2008) to create \emph{more},
not \emph{less}.

Why is it important to protect the commitment part of a creative work
then? Because it creates a healthy market in which creators compete with
each other fairly, contributing more and more excellent works. There is
an implicit selection process, just like natural selection that selected
\emph{Homo sapiens}, that picks out the best creators. When we are
reading a novel, the joy comes from not only the stirring story
\emph{per se}; it also comes from the enormous hours the writer spent in
the novel, and more importantly, in improving her writing skills. The
work is a reflection of the author's skills and hard work. Evidently,
the more welcome a work is, the more likely the author will create
excellent work in the future. In \emph{The Plagiarist's Tale}, Rowan did
a ``literature mashup'', copying words from other spy novels and made
his own one (Widdicombe 53). What he did was more than stealing others'
works to gain revenue for himself: he tricked readers into thinking that
he was a capable reader who would produce fantastic novels. When the
misconduct came to the light, readers naturally felt extremely angry, as
their expectations were shattered. Plagiarism of this type must be
eliminated. To greedy writers, unpunished plagiarism will encourage them
to do the same thing; to writers who genuinely deserve acclaims, their
interests will be divided by those plagiarists and they will have less
incentive to create new works. The market could only afford a fixed
amount of novel writers. When a fake author gain a position in the pool,
some potential writer must lose his or her chance, and we might lose a
great novel. Some argued that in the field of music, composers also
copied others' pieces of music but their works are nonetheless welcome
(Gaylor 2008). However, these notes are more like ideas I discussed
above; they are too ordinary to count as a musician's commitment
(Gladwell 121). The hardest part of creating a music is putting a piece
of music inside a bigger part of music in a harmony way and playing the
whole work appropriately. Even if a band borrows others' work, audience
can still evaluate its members' skills accurately. This is like
attending a concert: we applaud the composer's genius as well as the
excellent interpretation by the skillful orchestra. To sum up, because
of our needs to differentiate more skillful creators from less skillful
ones, we have to protect authors' rights over their intellectual
property. Widdicombe called plagiarism as an ethical offense (Widdicombe
114), but I think that the reason why we don't have a law against
plagiarism itself is that the ethical offense plays the role of a common
law.

Some people will not be convinced by the above arguments and claim that
what we need is a law against plagiarism rather than a copyright law. If
copyright is all about a fair environment, the society can simply check
if an author credit all sources appropriately. However, in the real
world, it's far from uncommon that we heard news where original authors
of a work fired a lawsuit against works derived from theirs. The reasons
can vary significantly. In \emph{Something Borrowed}, Gladwell recounted
the story of an expert named Dorothy Lewis, who investigated in serial
killers for a long time, found an opera \emph{Frozen} almost replicated
her life story. She felt extremely uncomfortable with it. It's like
having thousands of people knowing every secret of her life.
Particularly, ``Lewis'' in the opera had an affair that never took place
in the real life. In \emph{RiP!: A Remix Manifesto}, Dan O'Neil also
told the story of how he strived to gain the ownership of his own
artwork (Gaylor 2008). One can also imagine a situation in which a cute
character is used without explicit permission by a company in its
advertisement, while the author of that character doesn't trust the
product of that company. There are so many harms that can be done to
authors if they don't formally own their creations. Even Lessig, one
major character of \emph{RiP!: A Remix Manifesto} who thinks our current
copyright law problematic, partially supported my concern in practice.
In his Creative Commons---a set of licenses that encourage distributing
creative works freely---has a special type of license that prohibits any
kind of derivational works. Without protecting a work's potential
income, only these artists interested in money would leave; without
copyright in general, even those dedicated to art will refuse to
contribute.

In conclusion, I think the protection of intellectual property is a
necessity, regardless the way we distribute and consume creative works.
The purpose of copyright is to encourage more creative works rather than
obstructing creativity. Ideas should not be protected, since it is
unhelpful, it not harmful, to a creative society. Instead, we should
protect the hard work of creators so that a fair environment for
creation can be promoted and creators can be protected from unexpected
harms.

\begin{references}
\item
  Widdicombe, Lizzie. ``The Plagiarist's Tale.'' \emph{The New Yorker}.
  Feb. 13 \& 20, 2012. Print.
\item
  Gladwell, Malcolm. ``Something Borrowed.'' \emph{The New Yorker}. 22
  Nov. 2004. Web. 10 May 2011.
\item
  Tomar, Dave. ``Use Me.'' \emph{The Shadow Scholar: How I Made a Living
  Helping College Kids Cheat}. New York: Bloomsbury, 2012. 79 -- 87.
  Print.
\item
  Gaylor, Brett, et al. ``RiP!: A Remix Manifesto.'' Web. 2008.
\end{references}

\end{document}
