\documentclass{writing}

\title{2A}
\date{02/02/16}

\begin{document}

\maketitle

The principle underlying Gladwell's article is about
\emph{creativity}.~As long as old pieces are served for new ideas, it's
\emph{borrowing}. If not, it's \emph{plagiarism}. The definition is
somehow deviated from the legal definition of intellectual property,
since what we currently have in law seems problematic to Gladwell. The
right of intellectual property, as Gladwell quoted from Lessig, is not a
property right in the traditional sense: by taking words or music pieces
from others one does not make the original authors worse off. In other
words, the copyable nature of intellectual property makes everything
tricky. The motivation behind protecting such right is to give economic
as well as prestige incentives to authors at the price of the
inconvenience of the whole society---people want more creativity.
However, as Gladwell pointed out, there is a balance between private
interests and the interests of the whole society, and when we give too
many preferences to private agents, it's likely that we might hinder
creativity. The premise of such claim's validity is that human beings
rely on previous works to create something new. The author is aware of
this premise and present numerous examples from music to literature. He
talked about his experience with a friend who showed him how similar
popular songs are to each other. He also mentioned his own case involved
with the opera of serial killer: the writer of the opera quoted
Gladwell's expression of ``a sin and a symptom'' without attribution,
but these very words belong to Gandhi, who was likely to copy it from
others in the literature history.

Gladwell, by sharing some ridiculous stories, gave a second point of his
principle. That is, we shall not avoid plagiarism only to avoid lawsuits
or scandals. On the one hand, he mentioned a professor accused of
plagiarism because of less than 20 words, a number insignificant
compared with the overall length of her work. On the other hand, he
talked that journalists, after paraphrasing, always checked carefully if
they got into the forbidden area accidentally. It's ironical that a
mostly original work is accused of plagiarism while, as Gladwell put,
``new words in the service of an old idea'', is a perfectly fine
practice.

Based on Gladwell's principle, the opera which borrowed from Gladwell's
article is mostly fine. This attitude was quite explicit. However, it
had side-effects on Dorothy Lewis that can not be ignored. In other
words, though we have a standard to judge whether one is plagiarizing,
the social issues brought by borrowing is not fully resolved. Does
copying an intellectual property really not make one worse off, as
Lessig proposed? For example, one's potential income from the future is
hampered. There is a lot of space for argument here.

Widdicombe had a much more traditional view of plagiarism, yet he did
have positive attitude towards creativity behind borrowing. To him,
plagiarism is copying without attributing to the original authors.
However, his sympathetic theme reflected his feelings. After all, Rowan
did compose a pretty good novel. At last, he compared plagiarism with
table-manner and argued that this is really an ethnic issue. He said
that there existed a lot of people who are willing to commit crimes but
not to copy from others, just like the very same people will not eat
half-eaten foods from others. By converting a legal issue into a moral
one, Widdicombe avoided the problem of defining a principle to judge
whether it is a plagiarism or a borrowing. Nevertheless, from the
communication e-mail he had quoted which said that Rowan was doubtlessly
a good editor, he would probably agree that when using materials created
by others, one should give credits to the author.

Since he really did not talk much about a principle, there is little to
comment. Still, one question remains open: under which circumstance
should we cite? If I write ``do or do not, there is no try'', I guess I
neither need to thank Master Yoda nor George Lucas, who in turn didn't
mention Shakespeare. Maybe Widdicombe would say it is also a moral
issue, but I believe at least there are cases in which we have to cite.

\end{document}
