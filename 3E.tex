\documentclass[12pt,letterpaper]{article}

\usepackage{ifpdf}
\usepackage{mla}

\begin{document}
\begin{mla}{Minsheng}{Liu}{Marissa Perino}{WCWP 10A}{03/02/16}
  {Untitled Essay}

Big data analytics has become an increasingly popular problem-solving
approach. Is this a good phenomenon? In academic researches, many people
have the concern that the nature of big data analytics---that it is more
about correlational descriptions than causal explanations---would have
negative impact to human being's pursuit of the explanation of the
universe. Unfortunately, supporters of big data sometimes ignore this
type of counterarguments. In the book \emph{Big Data}, Viktor
Mayer-Schönberger and Kenneth Cuvier concentrates on the applications
and ramifications of big data in the industry rather than the academe.
Cesar Hidalgo, though being a professor at MIT, refutes critics of this
sort by asking them to see big data's immediate benefits instead of
responding them directly. I hope this essay can handle this question by
showing that big data analytics does not necessarily hinder researchers'
pursuit for causal explanations. Results gained from big data analytics
are sometimes indispensable for explanatory theories, and such
techniques can be handy for many researchers.

Everyone knows that big data analytics can only reveal correlations, not
causal explanations. Then, there is a common opposition against it: this
technology will ``distract'' researchers' focus from explaining our
world causally. This does not hold. First of all, researchers are
probably the group who loves causal explanations most. When they have to
publish a result that only reveals some correlation about a subject,
that subject must be nontrivial. However, some progress is better than
no progress. If a researcher decides not to publish anything until he or
she can give a convincing theory, other researchers will be discouraged
from solving that problem, and grants will likely to fly away. From this
perspective, applying big data analytics to a field is good for the long
term development in that field. Second, big data analytics is not an
exclusive mythology; it is a tool meant to complement other methods.
Continue the above example. After a researcher uses big data techniques
to publish some descriptive results, he or she might continue in-depth
analysis---after gaining more grants---to obtain some explanatory
theories. Next, the academe has a series of efficient institutions in
ensuring rigor researches, like peer reviews during the publication
process. These institutions can prevent researchers from deviating from
causal studies. Last but not the least, the whole society has already
developed a strong attachment to causal explanations. This attachment
has already existed when Galileo and his peers started modern science by
using quantitative method. In fact, it can be traced back to Aristotle's
formal study of logic. Yet what matters is that stories of those ancient
thinkers are told over and over again to every child. Thus, it is
impossible for us to give up our pursuit for causal explanations.

There is an even tricker counterargument. Applying big data analytics to
traditional subjects of science yields many new subjects. For example,
machine learning and statistical methods have been two important
techniques of natural language processing, an interdisciplinary branch
developed from linguistics. In spite of those new subjects' real world
applications, do they have theoretical contributions to the original
fields? If not, is this a waste of human resources, considering the
possible breakthroughs in those fields? To understand this question, one
can look at the example of natural language processing more closely.
Machine translation is currently done by feeding machines with large
corpus. Machine learning algorithms can automatically figure out some
enigmatic patterns underlying human languages, possibly involving
millions of variables. Substantial progress has been made using this
approach. Yet a theoretical linguist would argue that such advance is
unhelpful for understanding how the brain works, which is the ultimate
goal of studying natural languages. Indeed, the models generated by
machine learning are meaningless to human beings---who can understand
formulas involving so many variables? Moreover, when some day we finally
understand the rules behind our language systems, will anyone be
interested in studying those old models? Therefore, a critic of big data
analytics can conclude that advances in statistical-based technologies
do not help us in understanding the world. Any research effort in these
fields, from a long point of view, is a waste of resources.

Nothing is further from the truth. The results obtained in fields
``generated'' by big data analytics---both experiences accumulated and
tools built---can be quite useful when researchers find explanatory
answers. We all know that it is Newton who discovered the gravity and
explained why our solar system rotates this way. Does this mean that
astronomical observations before Newton was useless? After all, with
Newton's discovery we could predict how a planet would move and generate
those observations identically. The answer is no, because Newton needed
these data to build a possible theory in the first place. After that,
Newton also needed other data to verify his theory. In a similar manner,
current works in machine translation could also be helpful in explaining
how brain works. Scientists cannot say the exact meaning of each
variable of a machine-learnt model, but linguists can still find some
general patterns about our language capacities. In the future, cognitive
scientists can use these experiences to narrow down their searches for
possible brain models. Moreover, language machines trained by large
corpus can serve as evaluation tools for explaining human brains.
Instead of setting up an expensive experiment with real human beings,
scientists can first verify their theories using machines. In short,
these seemingly \emph{ad hoc} fields---those generated by big data
analytics---are actually prerequisites in finding causal explanations.
Without efforts in these fields, finding a good explanation would be, if
not impossible, much harder.

Both arguments above separate correlational descriptions predicted by
big data analytics from causal explanations, but in fact, big data
analytics can \emph{directly} facilitate researchers in discovering
causal explanations. The idea is best characterized by a simple example
in math. Mathematically, a function is a procedure that given some input
yields some output. When analyzing the behavior of a function,
mathematicians often plot the graph of that function, even though such a
figure cannot constitute any formal proof that can be published. Why do
mathematicians still do this? They plot functions because from these
graphs, mathematicians can have a general understanding of functions in
question in order to pick the right tools. It is interesting that in
many cases, these functions are too complex to be plotted by hand.
Mathematicians have to use a computer to draw graphs. How similar is
this to big data! It is not hard to conceive that today's social
scientists will face data so complex that they cannot analyze by hand.
They need big data analytics to have some research directions before
conducting any manual analysis. Moreover, social scientists might be
able to ask the machine to find interesting phenomena automatically.
Only then do they conduct any in-depth case studies. I came up with both
examples within a relatively short period of time. It is therefore
reasonable for me to believe that, as techniques harnessing big data
become more and more sophisticated, the productivity of researchers will
increase even more.

\end{document}
