\documentclass[12pt,letterpaper]{article}

\usepackage{ifpdf}
\usepackage{mla}

\begin{document}
\begin{mla}{Minsheng}{Liu}{Marissa Perino}{WCWP 10A}{02/29/16}{3C Reflection}

Several points are developed throughout the preparation of the
presentation and the class discussion itself. They greatly help me
articulate my own argument for 3D claim.

First, after I carefully examined the chapter \emph{Next}, I realized
that the fundamental difference brought by big data technology. In the
``ancient time'', people did not have enough data. They could only
observe the world at one or several points locally. There might be
several parameters that resulted in their observations but that were not
present globally. However, people did not which parameters mattered.
Consequently, people started to find the causation, hoping that the
model they concluded from observations would hold universally. The
situation no longer holds in the age of big data. This enables me to
defend my argument from those emphasizing the importance of causation:
the environment has changed, and now we can give up some old mindset.

Another point which I had developed and which I strengthened in the
class discussion was on the interaction between big data and causation
discovery. The question asked in class was similar to that how people
innovation played within the big data challenges. My response was
organized around the idea that big data is not an exclusive tool. In
some situations, by choosing one particular strategy he or she give ups
the other choice. For example, a nation-state cannot be both autocratic
and democratic. However, big data are more like a complementary tools.
For instance, when finding properties of a set of math equations, it is
helpful to visualize them. This does not mean that the mathematician
would stop there and claim certain properties directly from the plot.
Instead, they will prove these properties nonetheless, but with the help
of this visualization, they can know what to prove in the first place
and they may have some ideas of how to prove from the figure. The
visualization here is like big data, and the processing power is
directly implemented in our brain. I believe this serves as a good
example of how big data interact with causation discovery, though I
would likely to conceive a more related example.

Last but not the least, a central point that was concluded by my
teammates is that we should put big data into practice now. Only by
repeated application of big data could we improve our understanding of
it. This helps me in formulating my main claim that it is indeed
necessary to accept a paradigm shift and that we should embrace the
world of correlation-as-first-strategy. A similar point was also
discussed in ``Saving Big Data from Big Mouths'' by Hidalgo Cesar.

\subsection*{Works Cited}
\bibent Mayer-Schönberger, Viktor and Cukier, Kenneth. ``Big Data.'' 2013. Print.

\bibent Hidalgo Cesar, ``Saving Big Data from Big Mouths.'' \textit{Scientific American} 2014. Web.

\end{mla}

\end{document}
