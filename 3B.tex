\documentclass{writing}

\title{3B}
\date{02/24/16}

\begin{document}

\maketitle

\textbf{Challenge one.} By increasingly relying on big data and focusing
on correlation, researchers may be discouraged from finding scientific
explanations and become descriptivists of facts. Would not this pose a
threat to human intellectuality?

First, there are fields in which we only have ``small data''. For
example, in field linguistics, language data are sometimes scarce,
especially for languages that have few speakers. Linguists have to rely
on the traditional analysis approach to describe grammars of these
languages. Second, in many fields only research on causes will be
accepted. A mathematician cannot, for instance, verify a theorem for all
numbers smaller than one million and declares that he or she
\emph{proves} the theorem. Therefore, one does not need to worry that
big data might pollute the research environment. Third, there are
hypotheses so complicated that we have no ways to find rigor and
conclusive evidence, like those in comparative politics and cognitive
science. In such situations, checking against big data can make these
hypotheses more reliable. After all, as said in both the book and the
article by Hidalgo, it is better to make some progress than nothing
(Mayer-Scheönberger and Cukier 195, Hidalgo 150).

In contrast, it is not unconceivable that using big data can actually
make research faster. A political scientist could use big data to
discover interesting patterns first before doing in-depth case study.
Cognitive scientists can do similar things, saving grants from
experiments that do not show much statistical differences.

\textbf{Challenge two.} If big data can predict our future actions, what
is the meaning of life?

There are two perspectives to this question. First, are long-term
predictions possible? No, because we can never gather enough information
or have enough computational resource to process them. What we are
making are approximations. For a system as complex as those using big
data, every small change in input will result in drastic difference in
output. Long-term predictions rely on so many variables, so people will
hardly be able to make progress in this field. The second aspect is
about short-term predictions. For example, if Alice knows that she will
have a bad result for tomorrow's ACT test, she would be extremely upset.
Notice that for most people this would not be an issue if the prediction
result is on the positive side. Nevertheless, with big data, Alice could
understand her issue much earlier so that the chance that she move
toward a positive future actually increases.

\textbf{Other materials.}

Why big data suddenly matters? In the past, people can only observe a
small amount of data. One can imagine a case where people need to
discover more about the world. To understand what will happen in some
place far far away, people need to ensure their conclusions---based on
their observations about the surrounding world---hold universally. This
is the beginning of causality and science. However, with the advent of
information era, people can now observe anywhere. People no longer need
to ensure their conclusions hold in a certain area, since they can
observe that area directly.

\begin{references}
\item
  Mayer-Schönberger, Viktor and Cukier, Kenneth. ``Big Data.'' 2013. Print.

\item
  Hidalgo Cesar, ``Saving Big Data from Big Mouths.'' \textit{Scientific American} 2014. Web.
\end{references}

\end{document}
